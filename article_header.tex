\makeatletter %Make @ valid


\@ifclasswith{article}{xe}{
  \usepackage{xeCJK}
  \setCJKmainfont[AutoFakeBold=true, AutoFakeSlant=true]{BiauKaiTC}
  \setCJKsansfont[AutoFakeBold=true, AutoFakeSlant=true]{NotoSansTC-Regular}
  \XeTeXlinebreaklocale "zh"
  \XeTeXlinebreakskip = 0pt plus 1pt
}{
  \usepackage{CJKutf8} %
}

%\usepackage{libertine} % Use the Libertine font
%\usepackage{charter} % Use the Charter font
\usepackage{microtype} % Slightly tweak font spacing for aesthetics

\usepackage{geometry}%
\geometry{
%  showframe,
  a4paper,
  left=2.54cm,
  right=2.54cm,
  top=3.17cm,
  bottom=3.17cm,
  headheight=25pt
}

\usepackage{titling}
\setlength{\droptitle}{-9em}
\renewcommand{\maketitlehookb}{\vspace{-2em}}
\renewcommand{\maketitlehookc}{\vspace{-2em}}
\renewcommand{\maketitlehookd}{\vspace{-2em}}

\usepackage{amsfonts} %
\usepackage{amsmath} %
\usepackage{amssymb} %
\usepackage{amsthm} %
\usepackage{array} %
\usepackage{caption} %caption
\usepackage{centernot} %not symbol
\usepackage[inline, shortlabels]{enumitem} %list
\usepackage{fancyhdr} %header and foot
\usepackage{graphicx} %graphic
\usepackage{imakeidx} %index table
  \makeindex[intoc, title=Index,
    options= -s /Users/jack/Documents/Latex_headers/index_style.ist]
  \indexsetup{othercode={\lhead{\emph{Index}}}}
\usepackage{hyperref} %hyperlink
\usepackage{import} %import
\usepackage{listings}
\usepackage[scr=rsfs]{mathalfa} %mathfont
\usepackage{mathdots} %dots
\usepackage{mathtools} %math symbol
\usepackage[framemethod=TikZ]{mdframed} %box
\usepackage{multicol}
\usepackage{nicematrix} %matrix
\usepackage{pgfplots} %graphmake
  \pgfplotsset{compat=newest, compat/show suggested version=false}
\usepackage{romannum} %romannumber
\usepackage{siunitx} %physical quantities
\usepackage{stmaryrd}
\usepackage{subcaption}
\usepackage{tabularx} %table
\usepackage{thmtools} %math thm
\usepackage{tikz} %graphmake
  \usetikzlibrary{cd}
\usepackage{tkz-euclide}
\usepackage[center]{titlesec}
\usepackage[normalem]{ulem} %underlining for emphasis
\usepackage{xcolor} %color
%\usepackage{xparse}
\usepackage[all]{xy} %Xy-pic diagram
%\usepackage{algorithm}
%\usepackage{algpseudocode}
%\usepackage{algpascal}
\usepackage[boxed, longend]{algorithm2e} %algorithm
  \SetKwInOut{Input}{Input}\SetKwInOut{Output}{Output}
  \SetKwProg{Fn}{Function}{\string:}{}

\usepackage{lipsum}

%titlesec
\renewcommand{\thesection}{\Roman{section}}
\titleformat{\section}
  {\centering\normalsize\scshape}%\bfseries\uppercase
  {\thesection. }
  {12pt}
  {}

%hdr
\pagestyle{fancy}
%\renewcommand{\chaptermark}[1]{\markboth{\chaptername\ \thechapter.\ #1}{}}
\renewcommand{\sectionmark}[1]{\markright{\thesection.\quad \textsc{#1}}}

\fancyhf{}
\fancyhead[L]{}%\leftmark
\fancyhead[C]{}%\textsc{\thetitle}
\fancyhead[R]{\rightmark}%\rightmark
\fancyfoot[L]{}
\fancyfoot[C]{\thepage}
\fancyfoot[R]{}%\leftmark


%thmtools
\mdfsetup{skipabove=1em, skipbelow=1em, innertopmargin=5pt, innerbottommargin=6pt}

\theoremstyle{plain}

\declaretheoremstyle[
  headfont=\rmfamily\bfseries\scshape,
  bodyfont=\rmfamily\mdseries,
  %headpunct=.,
  %notebraces={(}{)},
  %postheadspace=\newline,
  mdframed={leftline=true, rightline=true, topline=true, bottomline=true}
]{framedaround}

\declaretheoremstyle[
  headfont=\rmfamily\bfseries\scshape,
  bodyfont=\rmfamily\mdseries,
  mdframed={leftline=true, rightline=false, topline=false, bottomline=false}
]{framedleft}

\declaretheoremstyle[
  headfont=\rmfamily\itshape,
  bodyfont=\rmfamily\mdseries,
  notebraces={ (}{)},
  headpunct={. },
  mdframed={skipabove=0em, leftline=true, rightline=false, topline=false, bottomline=false},
  qed={/\!/\!/\!/}
]{proofbox}

\declaretheoremstyle[
  headfont=\rmfamily\itshape,
  bodyfont=\rmfamily\mdseries,
  notebraces={ (}{)},
  headpunct=.,
  mdframed={skipabove=0em, leftline=true, rightline=false, topline=false, bottomline=false}
]{solbox}

\declaretheoremstyle[%
  %spaceabove=1em,
  headfont=\rmfamily\bfseries,%
  bodyfont=\rmfamily\mdseries,%
  headpunct=.,
  %mdframed={skipabove=0em, leftline=false, rightline=false, topline=false, bottomline=false}
]{defaultstyle}

\declaretheoremstyle[%
  %spaceabove=1em,
  headfont=\rmfamily\itshape,%
  bodyfont=\rmfamily\mdseries,%
  %mdframed={skipabove=0em, leftline=false, rightline=false, topline=false, bottomline=false}
]{defaultitstyle}

\declaretheorem[style=defaultitstyle, numbered=no, name=Definition]{defi}
\declaretheorem[style=defaultitstyle, numbered=no, name=Theorem]{thm}
\declaretheorem[style=defaultitstyle, numbered=no, name=Lemma]{lem}
\declaretheorem[style=defaultitstyle, numbered=no, name=Corollary]{cor}
\declaretheorem[style=defaultitstyle, numbered=no, name=Proposition]{prop}
\declaretheorem[style=defaultstyle, numbered=no, name=Remark]{rmk}
\declaretheorem[style=framedaround, numbered=no, name=Notation]{notation}
\declaretheorem[style=defaultstyle, numbered=no, name=Agreement]{agt}
\declaretheorem[style=defaultstyle, numbered=no, name=Fact]{fact}
\declaretheorem[style=defaultstyle, numbered=no, name=Example]{eg}
\declaretheorem[style=defaultstyle, numbered=yes, name=Exercise]{ex}
\declaretheorem[style=defaultstyle, numbered=yes, name=Problem]{pb}
\declaretheorem[style=proofbox, numbered=no, name=Proof]{pf}
\declaretheorem[style=defaultstyle, numbered=no, name=Solution]{sol}
\declaretheorem[style=defaultitstyle, numbered=no, name=Answer]{ans}
\declaretheorem[style=framedleft, numbered=no, name=Question]{question}
\declaretheorem[style=proofbox, numbered=no, name=Sketch]{sketch}
\declaretheorem[style=defaultstyle, numbered=no, name=Algorithm]{algo}


%list
%\setlist{leftmargin=1.5\parindent}


%math
% letters
\newcommand{\A}{\mathcal{A}}
\newcommand{\B}{\mathcal{B}}
\newcommand{\C}{\mathbb{C}}
\newcommand{\F}{\mathcal{F}}
%\newcommand{\I}{\mathcal{I}}
\newcommand{\N}{\mathbb{N}}
\renewcommand{\P}{\mathbb{P}}
\newcommand{\Q}{\mathbb{Q}}
\newcommand{\M}{\textsf{M}}
\newcommand{\R}{\mathbb{R}}
\newcommand{\T}{\mathbb{T}}
\newcommand{\U}{\mathcal{U}}
\newcommand{\V}{\mathcal{V}}
\newcommand{\X}{\mathcal{X}}
\newcommand{\Z}{\mathbb{Z}}
%\renewcommand{\H}{\mathbb{H}}
\newcommand{\K}{\mathcal{K}}
\renewcommand{\L}{\mathcal{L}}
\renewcommand{\S}{\mathcal{S}}
\renewcommand{\Re}{\mathfrak{Re}}
\renewcommand{\Im}{\mathfrak{Im}}


%operators
\DeclareMathOperator{\arsinh}{arsinh}
\DeclareMathOperator{\arcosh}{arcosh}
\DeclareMathOperator{\artanh}{artanh}
\DeclareMathOperator{\arcoth}{arcoth}
\DeclareMathOperator{\arsech}{arsech}
\DeclareMathOperator{\arcsch}{arcsch}

\DeclareMathOperator{\adj}{\mathsf{adj}}
\DeclareMathOperator{\Ann}{\mathsf{Ann}}
\DeclareMathOperator{\area}{area}
\DeclareMathOperator{\argmin}{arg min}
\DeclareMathOperator{\argmax}{arg max}
\DeclareMathOperator{\Aut}{Aut}
\DeclareMathOperator{\card}{card}
\DeclareMathOperator{\ccl}{ccl}
\DeclareMathOperator{\closure}{cl}
\DeclareMathOperator{\colspace}{\mathsf{C}}
\DeclareMathOperator{\compl}{comp}
\DeclareMathOperator{\Cond}{\mathsf{Cond}}
\let\det\relax
\DeclareMathOperator{\det}{\mathsf{det}}
\DeclareMathOperator{\diag}{diag}
\DeclareMathOperator{\ev}{\mathsf{ev}}
\DeclareMathOperator{\End}{End}
\DeclareMathOperator{\Gal}{Gal}
\DeclareMathOperator{\Hom}{Hom}
\DeclareMathOperator{\id}{\mathsf{id}}
\DeclareMathOperator{\im}{Im}
\DeclareMathOperator{\interior}{int}
\DeclareMathOperator{\irr}{irr}
\let\ker\relax
\DeclareMathOperator{\ker}{Ker}
\DeclareMathOperator{\lcm}{lcm}
\DeclareMathOperator{\nullspace}{\mathsf{N}}
\DeclareMathOperator{\nullity}{\mathsf{nullity}}
\DeclareMathOperator{\ord}{ord}
\DeclareMathOperator{\powerset}{\mathcal{P}}
\DeclareMathOperator{\proj}{\mathsf{proj}}
\DeclareMathOperator{\range}{\mathsf{R}}
\DeclareMathOperator{\rank}{\mathsf{rank}}
\DeclareMathOperator{\sgn}{sgn}
\DeclareMathOperator{\spn}{\mathsf{span}}
\DeclareMathOperator{\sym}{Sym}
\DeclareMathOperator{\trace}{\mathsf{trace}}


%Vector
\DeclareDocumentCommand{\vectorbold}{ s m }{\IfBooleanTF{#1}{\boldsymbol{#2}}{\mathbf{#2}}} % Vector bold [star for Greek and italic Roman]
\DeclareDocumentCommand{\vb}{}{\vectorbold} % Shorthand for \vectorbold

\DeclareDocumentCommand{\vectorarrow}{ s m }{\IfBooleanTF{#1}{\vec{\boldsymbol{#2}}}{\vec{\mathbf{#2}}}} % Vector arrow + bold [star for Greek and italic Roman]
\DeclareDocumentCommand{\va}{}{\vectorarrow} % Shorthand for \vectorarrow

\DeclareDocumentCommand{\vectorunit}{ s m }{\IfBooleanTF{#1}{\boldsymbol{\hat{#2}}}{\mathbf{\hat{#2}}}} % Unit vector [star for Greek and italic Roman]
\DeclareDocumentCommand{\vu}{}{\vectorunit} % Shorthand for \vectorunit

\DeclareDocumentCommand{\dotproduct}{}{\boldsymbol\cdot} % Vector dot product symbol
\DeclareDocumentCommand{\vdot}{}{\dotproduct} % Shorthand for \dotproduct [note that the command sequence \dp is protected]


% brackets and braces
\DeclareDocumentCommand\quantity{}{{\ifnum\z@=`}\fi\@quantity}
\DeclareDocumentCommand\@quantity{ t\big t\Big t\bigg t\Bigg g o d() d|| }
{ % Flexible automatic bracketing of an expression in () or [] or {} or ||
  % Handles manual override of sizing
  \IfBooleanTF{#1}{\let\ltag\bigl \let\rtag\bigr}{
    \IfBooleanTF{#2}{\let\ltag\Bigl \let\rtag\Bigr}{
      \IfBooleanTF{#3}{\let\ltag\biggl \let\rtag\biggr}{
        \IfBooleanTF{#4}
        {\let\ltag\Biggl \let\rtag\Biggr}
        {\let\ltag\left \let\rtag\right}
      }
    }
  }
  % Handles actual bracketing
  \IfNoValueTF{#5}{
    \IfNoValueTF{#6}{
      \IfNoValueTF{#7}{
        \IfNoValueTF{#8}
        {()}
        {\ltag\lvert{#8}\rtag\rvert}
      }
      {\ltag(#7\rtag) \IfNoValueTF{#8}{}{|#8|}}
    }
    {\ltag[#6\rtag] \IfNoValueTF{#7}{}{(#7)} \IfNoValueTF{#8}{}{|#8|}}
  }
  {\ltag\lbrace#5\rtag\rbrace  \IfNoValueTF{#6}{}{[#6]} \IfNoValueTF{#7}{}{(#7)} \IfNoValueTF{#8}{}{|#8|}}
  \ifnum\z@=`{\fi}
}
\DeclareDocumentCommand\qty{}{\quantity} % Shorthand for \quantity
\DeclareDocumentCommand\pqty{ l m }{\braces#1{\lparen}{\rparen}{#2}}
\DeclareDocumentCommand\bqty{ l m }{\braces#1{\lbrack}{\rbrack}{#2}}
\DeclareDocumentCommand\Bqty{ l m }{\braces#1{\lbrace}{\rbrace}{#2}}
\DeclareDocumentCommand\vqty{ l m }{\braces#1{\lvert}{\rvert}{#2}}

\DeclareDocumentCommand\pmqty{m}{\begin{pmatrix}#1\end{pmatrix}}
\DeclareDocumentCommand\Pmqty{m}{\left\lgroup\begin{matrix}#1\end{matrix}\right\rgroup}
\DeclareDocumentCommand\bmqty{m}{\begin{bmatrix}#1\end{bmatrix}}
\DeclareDocumentCommand\vmqty{m}{\begin{vmatrix}#1\end{vmatrix}}
\DeclareDocumentCommand\matrixquantity{}{{\ifnum\z@=`}\fi\@matrixquantity}
\DeclareDocumentCommand\@matrixquantity{ s g o d() d|| }{
  \mathord{
    \IfNoValueTF{#2}{
      \IfNoValueTF{#3}{
        \IfNoValueTF{#4}{
          \IfNoValueTF{#5}{()}{
            \vmqty{#5}
          }
        }{
          \IfBooleanTF{#1}{\Pmqty{#4}}{\pmqty{#4}}
          \IfNoValueTF{#5}{}{|#5|}
        }
      }{
        \bmqty{#3} \IfNoValueTF{#4}{}{(#4)} \IfNoValueTF{#5}{}{|#5|}
      }
    }{
      \begin{matrix}#2\end{matrix} \IfNoValueTF{#3}{}{[#3]} \IfNoValueTF{#4}{}{(#4)} \IfNoValueTF{#5}{}{|#5|}
    }
  }
  \ifnum\z@=`{\fi}
}
\DeclareDocumentCommand\mqty{}{\matrixquantity} % Shorthand for \matrixquantity
\DeclareDocumentCommand\matrixdeterminant{m}{\vmqty{#1}} % Matrix determinant
\DeclareDocumentCommand\mdet{}{\matrixdeterminant} % Shorthand for matrix determinant

\DeclareDocumentCommand\spmqty{m}{\pqty{\begin{smallmatrix}#1\end{smallmatrix}}}
\DeclareDocumentCommand\sPmqty{m}{\left\lgroup\begin{smallmatrix}#1\end{smallmatrix}\right\rgroup}
\DeclareDocumentCommand\sbmqty{m}{\bqty{\begin{smallmatrix}#1\end{smallmatrix}}}
\DeclareDocumentCommand\svmqty{m}{\vqty{\begin{smallmatrix}#1\end{smallmatrix}}}
\DeclareDocumentCommand\smallmatrixquantity{ s g o d() d|| }{
  \mathord{
    \IfNoValueTF{#2}{
      \IfNoValueTF{#3}{
        \IfNoValueTF{#4}{
          \IfNoValueTF{#5}{()}{
            \svmqty{#5}
          }
        }{
          \IfBooleanTF{#1}{\sPmqty{#4}}{\spmqty{#4}}
          \IfNoValueTF{#5}{}{|#5|}
        }
      }{
        \sbmqty{#3} \IfNoValueTF{#4}{}{(#4)} \IfNoValueTF{#5}{}{|#5|}
      }
    }{
      \begin{smallmatrix}#2\end{smallmatrix} \IfNoValueTF{#3}{}{[#3]} \IfNoValueTF{#4}{}{(#4)} \IfNoValueTF{#5}{}{|#5|}
    }
  }
}
\DeclareDocumentCommand\smqty{}{\smallmatrixquantity} % Shorthand for \smallmatrixquantity
\DeclareDocumentCommand\smallmatrixdeterminant{m}{\svmqty{#1}} % Small matrix determinant
\DeclareDocumentCommand\smdet{}{\smallmatrixdeterminant} % Shorthand for small matrix determinant
\DeclareDocumentCommand\evaluated{ s g d[| d(| }
{ % Vertical evaluation bar
	\IfNoValueTF{#2}
	{
		\IfNoValueTF{#3}
		{
			\IfNoValueTF{#4}
				{\argopen.\vphantom{\int}\argclose\rvert}
				{\IfBooleanTF{#1}{\vphantom{#4}}{}\left(\IfBooleanTF{#1}{\smash{#4}}{#4}\vphantom{\int}\right\rvert}
		}
		{\IfBooleanTF{#1}{\vphantom{#3}}{}\left[\IfBooleanTF{#1}{\smash{#3}}{#3}\vphantom{\int}\right\rvert \IfNoValueTF{#4}{}{(#4|}}
	}
	{\IfBooleanTF{#1}{\vphantom{#2}}{}\left.\IfBooleanTF{#1}{\smash{#2}}{#2}\vphantom{\int}\right\rvert \IfNoValueTF{#3}{}{[#3|} \IfNoValueTF{#4}{}{(#4|}}
}
\DeclareDocumentCommand\eval{}{\evaluated} % Shorthand for evaluated

\DeclarePairedDelimiterX{\abs}[1]{\lvert}{\rvert}{\ifblank{#1}{\:\cdot\:}{#1}}
\DeclarePairedDelimiterX{\braket}[1]{\langle}{\rangle}{\ifblank{#1}{\:\cdot\:}{#1}}
\DeclarePairedDelimiterX{\floor}[1]{\lfloor}{\rfloor}{\ifblank{#1}{\:\cdot\:}{#1}}
\DeclarePairedDelimiterX{\ceiling}[1]{\lceil}{\rceil}{\ifblank{#1}{\:\cdot\:}{#1}}
\DeclarePairedDelimiterX{\norm}[1]{||}{||}{\ifblank{#1}{\:\cdot\:}{#1}}
\DeclarePairedDelimiterX{\innerp}[2]{\langle}{\rangle}{%
  \ifblank{#1}{\:\cdot\:}{#1},%
  \ifblank{#2}{\:\cdot\:}{#2}%
}

\providecommand{\given}{}%
\newcommand{\SetSymbol}[1][]{%
  \nonscript\:#1\vert%
  \allowbreak%
  \nonscript\:%
  \mathopen{}%
}%
\DeclarePairedDelimiterX{\Set}[1]{\{}{\}}{
  \renewcommand{\given}{\SetSymbol[\delimsize]}#1%
}

%tikz
\pgfarrowsdeclarecombine{twolatex'}{twolatex'}{latex'}{latex'}{latex'}{latex'}
\tikzset{->/.style = {decoration={markings,
                                  mark=at position 1 with {\arrow[scale=2]{latex'}}},
                      postaction={decorate}}}
\tikzset{<-/.style = {decoration={markings,
                                  mark=at position 0 with {\arrowreversed[scale=2]{latex'}}},
                      postaction={decorate}}}
\tikzset{<->/.style = {decoration={markings,
                                   mark=at position 0 with {\arrowreversed[scale=2]{latex'}},
                                   mark=at position 1 with {\arrow[scale=2]{latex'}}},
                       postaction={decorate}}}
\tikzset{->-/.style = {decoration={markings,
                                   mark=at position #1 with {\arrow[scale=2]{latex'}}},
                       postaction={decorate}}}
\tikzset{-<-/.style = {decoration={markings,
                                   mark=at position #1 with {\arrowreversed[scale=2]{latex'}}},
                       postaction={decorate}}}
\tikzset{->>/.style = {decoration={markings,
                                  mark=at position 1 with {\arrow[scale=2]{latex'}}},
                      postaction={decorate}}}
\tikzset{<<-/.style = {decoration={markings,
                                  mark=at position 0 with {\arrowreversed[scale=2]{twolatex'}}},
                      postaction={decorate}}}
\tikzset{<<->>/.style = {decoration={markings,
                                   mark=at position 0 with {\arrowreversed[scale=2]{twolatex'}},
                                   mark=at position 1 with {\arrow[scale=2]{twolatex'}}},
                       postaction={decorate}}}
\tikzset{->>-/.style = {decoration={markings,
                                   mark=at position #1 with {\arrow[scale=2]{twolatex'}}},
                       postaction={decorate}}}
\tikzset{-<<-/.style = {decoration={markings,
                                   mark=at position #1 with {\arrowreversed[scale=2]{twolatex'}}},
                       postaction={decorate}}}

\tikzset{circ/.style = {fill, circle, inner sep = 0, minimum size = 3}}
\tikzset{scirc/.style = {fill, circle, inner sep = 0, minimum size = 1.5}}
\tikzset{mstate/.style={circle, draw, blue, text=black, minimum width=0.7cm}}

\tikzset{eqpic/.style={baseline={([yshift=-.5ex]current bounding box.center)}}}
\tikzset{commutative diagrams/.cd,cdmap/.style={/tikz/column 1/.append style={anchor=base east},/tikz/column 2/.append style={anchor=base west},row sep=tiny}}

\definecolor{mblue}{rgb}{0.2, 0.3, 0.8}
\definecolor{morange}{rgb}{1, 0.5, 0}
\definecolor{mgreen}{rgb}{0.1, 0.4, 0.2}
\definecolor{mred}{rgb}{0.5, 0, 0}


%Others
\newcommand{\littletaller}{\mathchoice{\vphantom{\big|}}{}{}{}}%
\newcommand{\restr}[2]{{%restriction
  \left.\kern-\nulldelimiterspace %resize the bar with \right
  #1 %
  \littletaller %
  \right|_{#2} % the delimiter
}}

\DeclareFontFamily{U}{tipa}{}
\DeclareFontShape{U}{tipa}{m}{n}{<->tipa10}{}
\newcommand{\arc@char}{{\usefont{U}{tipa}{m}{n}\symbol{62}}}%
\newcommand{\arc}[1]{\mathpalette\arc@arc{#1}}
\newcommand{\arc@arc}[2]{%
  \sbox0{$\m@th#1#2$}%
  \vbox{
    \hbox{\resizebox{\wd0}{\height}{\arc@char}}
    \nointerlineskip
    \box0
  }%
}

%\renewcommand{\subset}{\subseteq}
%\renewcommand{\supset}{\supseteq}
\newcommand{\nimplies}{\centernot\implies}
\newcommand{\nimpliedby}{\centernot\impliedby}
\renewcommand{\nmid}{\centernot\mid}
\newcommand{\contradiction}{\rightarrow\!\leftarrow}
\newcommand{\st}{s.t.\ }
\newcommand{\Cdot}{\:\cdot\:}
\newcommand{\Vdot}{\:\vdot\:}
\newcommand{\Pmod}[1]{\ (\operatorname{mod}\,#1)}


%others
\linespread{1.36}
\renewcommand{\thefootnote}{\arabic{footnote}}


\makeatother